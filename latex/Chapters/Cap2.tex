%************************************************
\chapter{Radianza, primo segnale di speranza}
\label{chp:cap2}
%************************************************

\`E l'anno 4692 quando Yaniel decide di annunciare pubblicamente il suo sdegno verso i crociati di Mendev. E non parliamo di una paladina qualsiasi, ma di un membro ufficiale dell'Ordine Brillante, nonch\'e conclamata cacciatrice di demoni. Yaniel, in un giorno di particolare ispirazione, decide infatti di manifestare ben pi\`u di un comune dissenso, rivolgendosi direttamente ai suoi commilitoni accusandoli di negligenza e poltroneria. La dichiarazione avviene a gran voce con tanto di foglio di pergamena, nel bel mezzo della piazza di Clydwell e alla non casuale portata delle vicinissime orecchie del prelato. Per la gente dell'epoca il riferimento di tali accuse \`e pi\`u che chiaro: \`e recentissima la ferita ancor pi\`u morale che fisica circa l'improvviso attacco di Khorramzadeh, meglio noto come il Signore delle Tempeste. Si narra infatti che poco prima che i difensori riuscissero a ricacciare lui e suoi famigli nell'Abisso, la Pietra Guardiana sia stata attaccata e persino scalfita da un fiammeggiante fendente dell'enorme demone. Sono dunque paura, incertezza e scoraggiamento i sentimenti che pesano sul morale dei cittadini di Kenabres, poich\'e stavolta neppure le poderose mura della loro amata citt\`a sono bastate a tenerli al sicuro.
Forse perch\`e le parole di Yaniel vanno troppo vicine alla realt\`a, la paladina viene velocissimamente sospesa dai suoi incarichi e direttamente scomunicata dal prelato. I suoi stessi superiori sono complici della severissima punizione. Disonorata e colta da mille emozioni Yaniel reagisce di istinto, lasciando la citt\`a decisa a riscattare se stessa tramite una santa e personale crociata. Assieme alla sua forza di volont\`a brandisce la sua clamorosa spada incantata, Radianza, conducendo la sua battaglia nel cuore del territorio dei demoni. Nei due anni successivi della paladina non si hanno pi\`u notizie e in Kenabres viene prontamente dichiarata dispersa.

\`E l'anno 4694 quando Yaniel viene avvistata in prossimit\`a della Porta Nord. Barcollante, fiera e con Radianza saldamente in cinta, marcia a capo di un piccolo esercito di crociati, liberati nel corso delle sue imprese. In quegli anni sia lei che i suoi ufficiali sono oramai persone diverse. Da parte sua Yaniel ha rinunciato a un po' del suo orgoglio in cambio della consapevolezza delle difficili decisioni che un \emph{leader} \`e talvolta costretto ad affrontare; mentre gli ufficiali in capo sono altres\`i pi\`u anziani, ma consapevoli che spesso la verit\`a \`e esattamente ci\`o che \`e necessario sentire.
Ahim\`e Yaniel viene assassinata prima della fine dell'anno, ad opera del lilitu Minagho, scovato dopo appena una settimana dall'inizio dell'ennesima spedizione. I seguaci di lei riescono fortunatamente a tornare in Kenabres, strappando Radianza dalle grinfie dell'Abisso ma perdendo traccia del corpo della prode guerriera, che purtroppo non viene pi\`u recuperato. La spada, come sensibile alla mancanza della sua eroina, cade in un'opaca oscurit\`a inibendo opacamente i suoi poteri. Nonostante ci\`o i crociati dichiarano la lama ``reliquia dell'Ordine Brillante'', esponendola in segno di rispetto nella prestigiosa vetrina della sede dell'Ordine: la Grigia Guarnigione.

A qualche mese dai nostri giorni Radianza viene saccheggiata da alcuni operatori occulti, probabilmente affiliati a una delle diverse organizzazioni criminali ormai radicate in citt\`a. Destino ha voluto che qualunque scellerato piano fosse riservato a Radianza non arrivasse mai a compimento, ma che piuttosto la mano di una nuova guerriera si stendesse sulla scintillante elsa dorata. Avventurieri, siete pronti ad entrare nella storia?