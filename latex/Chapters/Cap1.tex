%************************************************
\chapter{La caduta di Kenabres}
\label{chp:cap1}
%************************************************

Per diverse settimane un'atmosfera di eccitazione si \`e costruita in Kenabres: il gran festival di Armasse \`e alle porte! La tradizione vuole che studenti, sacerdoti e comuni cittadini si incontrino per studiare assieme lezioni di storia, sponsorizzando l'addestramento di giovani reclute, la nomina di scudieri e l'ordinamento di nuovi membri del clero. Con il passare del tempo il festival \`e cresciuto fino ad includere spettacoli teatrali, duelli di spada e persino vere e proprie giostre. In Kenabres l'evento \`e molto sentito, poich\'e occasione di svago in duri tempi di guerra. E sebbene tutta la cittadinanza sia coinvolta, il suo fulcro viene spesso ospitato nella piazza di Clydwell, sede della maestosa e omonima cattedrale.

La giornata di inaugurazione \`e per\`o drammatica. Nel momento stesso in cui Lord Hulrun in persona, inquisitore e prelato di Kenabres, sale all'abitacolo schiarendosi la voce, tutto un tratto rumori assordanti riempiono l'aria. Un vuoto che toglie il fiato premunisce il disastro: fragorosi crolli e mattoni sbriciolati fanno tremare la terra, mentre urla di ogni genere si alzano dal pienone della folla. Verso ovest, dove prima torreggiava il maschio della citt\`a, nonch\'e sede della Pietra Guardiana, \`e il nulla. Al suo posto un brillante incendio rosso, colmo di fulmini e polvere, esplode verso il cielo.

Il fato vuole che cinque avventurieri riprendano i sensi nella pi\`u totale oscurit\`a, coperti di macerie e polvere tutt'intorno. Seshilia, Feanor ``il Bello'', Lady Oscar, Gavin Denqueen e Fratello Morrog sono rimasti miracolosamente illesi da una caduta di decine di metri. Non appena le incessanti emicranie lasciano posto alla lucidit\`a, memorie del recente accaduto riaffiorano alla mente: ricordi di un drago d'argento, sfidato da un enorme demonio armato di spada fiammeggiante, si associano all'identit\`a di Terendelev, antico guardiano di Kenabres. La battaglia \`e frenetica, ma breve. Gli ultimi istanti di Terendelev si consumano tra una caduta rovinosa sulla facciata della cattedrale e il fendente mortale del Signore delle Tempeste. In quello stesso momento, un titanico e vermiforme demone erutta nel bel mezzo della piazza, riducendo in rovine diversi edifici nella zona. Un crepaccio creato dal rovinoso evento non lascia scampo, risucchiando in un vero e proprio abisso i malcapitati piedi dei cinque.

Per identit\`a di scopi il gruppo si fa strada tra i meandri delle profondit\`a, trovando e scortando tre sopravvissuti piuttosto eccentrici: Anevia, una donna la cui gamba \`e rimasta spezzata; Aravashnial, un mistico elfo rimasto cieco dalla recente caduta; e Horgus, un nobile arrogante disposto a pagare chiunque purch\'e lo porti fuori dai sotterranei. Nel tentativo di fare squadra si susseguono momenti di aiuto reciproco, ma anche di tensione, nonch\'e artistiche ed efficaci mediazioni. L'incontro con una giovane paladina, Amanil, avviene invece in un inaspettato santuario di Torag, costruito nel bel mezzo del nulla tra caverne oscure e strane bestie volanti.

Aiutata dal ritrovamento di una sala vecchia di decine di anni, una storia di deformi uomini, risalenti alla Prima Crociata, rialeggia tra i compagni, diventando realt\`a all'incontro di tre individui decisamente fuori dalla norma. Tra di loro un tizio di nome Lann, pi\`u simile a una salamandra che a un ragazzo, conduce tutti a Nethholm, sede della comunit\`a dei ``mutati''. Umido, fanghiforme e vagamente accogliente, l'insediamento \`e presidiato da un grande e voluminoso regnante: Re Sull. Persona mite, ma saggia, Re Sull condivide con gli avventurieri alcune vicissitudini locali, citando piccole trib\`u ribelli che la gente del posto chiama ``i traditori''. \`E solo dopo aver approfittato della calda ospitalit\`a offertagli che il gruppo decide di proseguire, tra il villaggio e la superficie rimane infatti un solo ostacolo: la tana dei traditori mutanti.