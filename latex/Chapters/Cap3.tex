%************************************************
\chapter{La fine dello scoiattolo}
\label{chp:cap3}
%************************************************

Al quindicesimo giorno dall'incursione dei demoni in Kenabres, il gruppo di avventurieri si trova nel Librarium Alanera, esattamente nella zona opposta rispetto a casa Tirabade. Quella stessa notte un ricordo lontano prende posto nel sogno di Anvenn. Viaggiando indietro nel tempo, fino ai tempi della sua infanzia, egli si rivede tra le strade di Nisroch, una citt\`a portuale dell'ovest, in prossimit\`a della Baia del Conquistatore. Alcune scene, della durata di pochi minuti, vengono vissute da Avenn come fossero reali, ma accellerate. Le prime immagini sono piacevoli, i pensieri lo portano a rivivere alcuni momenti con i compagni di gilda. Ladri, fuorilegge e borseggiatori, ma anche filosofi, oratori e musicisti erano membri usuali della vecchia compagnia. Un attimo dopo risuonano rumori di passi, come in una marcia. Nella mente di Anvenn vengono visualizzati uomini in sandali, vestiti in tunica bianca e armati di pugni di ferro. Sono gli agenti di una societ\`a chiamata ``Sindone Silenziosa'', il giorno in cui il crudele governatore da inizio ad una folle caccia all'uomo.

In quell'epoca \`e una sacerdotessa di Desna, Veeruh, che gli salva la vita portandolo in Kenabres e affidandolo alla cura di una giovane e promettente iniziata: Irabeth Tirabade. \`E per motivi di sicurezza che Anvenn da quel giorno in avanti sar\`a conosicuto come Anevia, un nome che nel folclore Varisiano viene associato scaramanticamente alla carta della fortuna, lo \emph{scoiattolo}. La sua vita, dopotutto, non era stata da meno.

Anevia era l\`i il giorno dell'ultimo festival di Armasse, il giorno dell'attacco dei demoni. Caduta e poi gravemente ferita \`e grazie ad un colpo di fortuna se riesce a sopravvivere: un coraggioso gruppo di avventurieri la trova esanime in una buia caverna, indifesa ma viva. Tuttavia presto o tardi gli interessi del gruppo si allontano dai suoi, la paura di non arrivare mai pi\`u a casa ha il sopravvento. Anevia, al quindicesimo giorno, decide di lasciare tutto e andare per la sua strada.

La sua scelta non \`e facile, n\'e gratuita. Ai membri pi\`u tenaci della compagnia, Gavin, Feanor e Seshilia \`e costretta a rivelare la sua identit\`a, ma \`e facendo leva sulla buon'anima di Amanil e Fratello Morrog che guadagna la  libert\`a di cercare il suo destino. Destino che si riveler\`a in fretta, come i suoi sogni: con passo lesto e qualche pozione di invisibilit\`a raggiunge casa, ma la scena che le si presenta \`e impietosa: muri gaffiati, silenzio assordante e puzzo di sangue accolgono il suo arrivo. \`E al suo primo passo oltre la soglia d'entrata che una voce grave e raschiante fa scattare la trappola: ``Sei dunque giunto, scoiattolino, il tuo cadavere sar\`a una dolce sorpresa per tua moglie''. Un mezzorco furfante di nome Vagrog, catturato anni prima da Irabeth nel corso di un arresto, ha scontato la sua pena ed \`e tornato dal suo esilio. Ci\`o nonostante la prigionia pare non abbia giovato al pericoloso criminale, che anzi \`e pronto a vendicarsi colpendo l'affetto pi\`u caro della sua arcinemica.

La storia di Anvenn o Anevia che dir si voglia finisce qui, stroncata da un magico tramortimento e un secondo prima che il suo cuore smetta di battere... 

Un piccolo \emph{crak} risuona all'istante dell'omicidio, una luce abbagliante irrompe dalle finestre e un calore avvolgente ma confortevole riempie la stanza. Una visione celestiale appare nella mente di tutte le donne e gli uomini nel raggio di diversi chilometri: una creatura umanoide vagamente attraente, adornata di corna di demonio e ali da pipistrello spunta evocata al cospetto di cinque condottieri in una sala circolare. La Pietra Guardiana non c'\`e pi\`u, distrutta in mille frammenti ha rilasciato una forma di energia cos\`i potente da richiamare Lady Vorlesh in persona, la fattucchiera delle recenti storie responsabile dell'invasione dei demoni nel nostro mondo. 

Come rinnovata da una forza dentro di lei, Anevia riprende conoscenza. Le sue mani sono bianche come la neve e i suoi capelli biondi come l'oro, due ali piumate le cingono la schiena e i fianchi. Le spoglie polverizzate di Vagrog le fanno compagnia sparse nel pavimento, nella strana ma autentica consapevolezza che lo stesso destino sia toccato a centinaia di agenti dell'Abisso cos\`i arroganti o sfortunati da trovarsi nel raggio di azione della Pietra. La fortuna dello scoiattolo forse \`e svanita, ma qualcosa di pi\`u grande ha riguadagnato il suo posto: la speranza.
