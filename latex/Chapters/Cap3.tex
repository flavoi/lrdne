%************************************************
\chapter{La fine dello scoiattolo}
\label{chp:cap3}
%************************************************

\`E nel Librarium Alanera che un ricordo lontano prende posto nel sogno di Anvenn. 

Viaggiando indietro nel tempo, fino ai tempi della sua infanzia, egli si rivede tra le strade della citt\`a portuale di Nisroch. Alcune scene, della durata di pochi minuti, si susseguono con velocit\`a crescente. Le prime immagini sono piacevoli, i pensieri gli fanno rivivere alcuni momenti con i compagni di gilda. Artisti, fuorilegge o borseggiatori professavano un mestiere rischioso ma si guadagnavano la pagnotta ogni giorno. \`E questione di attimi prima che la scena cambi. In fretta e furia nella testa di Anvenn risuonano rumori di passi, come di una marcia. Sono gli agenti di una societ\`a chiamata ``Sindone Silenziosa'', il giorno dell'occupazione. Anvenn si risveglia di soprassalto, in mente ha un indirizzo e la criptica frase di addio di sua madre: ``sono una rosa mezza appassita''. In quell'epoca \`e una sacerdotessa di Desna, Veeruh, che gli salva la vita. Per motivi di sicurezza Anvenn cambia identit\`a: d'ora in avanti sar\`a Anevia, un nome che nel folclore Varisiano viene associato scaramanticamente alla carta della fortuna, lo \emph{scoiattolo}. La sua vita, dopotutto, non era stata da meno.

Anevia era l\`i il giorno dell'ultimo festival di Armasse, il giorno dell'attacco dei demoni. Caduta e poi gravemente ferita \`e grazie ad una sorta di colpo di fortuna se riesce a sopravvivere. Un coraggioso gruppo di avventurieri la salva nuovamente, come prima le era successo in occasione di un gruppo di briganti, quando sua moglie Irabeth l'aveva tratta in salvo. Tuttavia presto o tardi gli interessi del gruppo si allontano dai suoi, la destinazione di casa le sembra sempre pi\`u distante e pertanto ancora una volta decide di lasciare tutto e andare per la sua strada. Ci\`o le \`e forse costato una piccolo rivelazione circa la sua vera identit\`a, ma da adesso \`e libera, come anni prima dopo l'arrivo in Kenabres, di cercare il suo destino. Destino che si compir\`a a breve: armata di furbizia e qualche pozione di invisibilit\`a raggiunge casa. La scena che le si presenta \`e per\`o impietosa: muri gaffiati, silenzio assordante e puzzo di sangue accolgono il suo arrivo. \`E al suo primo passo oltre la soglia d'entrata che una voce grave e raschiante accoglie il suo arrivo: ``Sei dunque giunto, scoiattolino, il tuo cadavere sar\`a una dolce sorpresa per tua moglie''. Un mezzorco furfante di nome Vagrog, sfuggito anni prima da una delle imprese di Irabeth, \`e tornato dal suo esilio, pronto a vendicarsi colpendo l'affetto pi\`u caro della sua arcinemica.

La storia di Anvenn o Anevia che dir si voglia finisce qui, stroncata da un magico tramortimento e a un secondo dalla pugnalata al cuore... Un piccolo \emph{crak} risuona nell'istante prima dell'omicidio, una luce abbagliante irrompe dalle finestre e un calore avvolgente ma confortevole avvolge la stanza. Una visione come un sogno ad occhi aperti accorre alla mente di tutti gli uomini nel raggio di diversi chilometri: una donna vagamente attraente, adornata di corna di demonio e ali da pipistrello spunta evocata al cospetto di cinque condottieri in una sala circolare. La Pietra Guardiana non c'\`e pi\`u, distrutta in mille frammenti ha rilasciato una forma di energia cos\`i potente da richiamare Lady Vorlesh in persona, la fattucchiera delle recenti storie responsabile dell'apertura della Piaga nel nostro mondo. 

Come rinnovata da una forza dentro di lei, Anevia riprende conoscenza. Le spoglie polverizzate di Vagrog le fanno compagnia sparse nel pavimento, nella strana ma autentica consapevolezza che lo stesso destino sia toccato a centinaia di agenti dell'Abisso cos\`i arroganti o sfortunati da trovarsi nel raggio di azione della Pietra. La fortuna dello scoiattolo forse ha colpito ancora, ma qualcosa di pi\`u grande ha riguadagnato il suo posto: la speranza.