%************************************************
\chapter*{Antefatto}
\pdfbookmark[1]{Antefatto}{Antefatto}
%************************************************

Per decenni i demoni hanno invaso, saccheggiato e soggiogato la regione. Temibili creature di ogni tipo, con pelle dura come il ferro, denti affilati come lame e occhi brucianti come fiamme, hanno vagato per queste terre devastate che un tempo erano conosciute come il florido regno di Sarkoris. Quattro crociate sono state indette dai capi della chiesa di Iomedae e quelli di diverse altre religioni, con lo scopo di ripulire la macchia nota come ``Piaga del Mondo'': un flagello cosmico di dimensioni chilometriche delimitato da fiamme nere. Pi\`u ci si avvicina alla Piaga del Mondo, pi\`u il mondo fisico diventa imprevedibile. Il terreno cambia davanti ai propri occhi, mutando forma con contorta deliberazione che sembra terrorizzare la terra stessa. Laide creature vengono vomitate dalla follia al centro del flagello, mostruosit\`a provenienti dalle profondit\`a dell'Abisso hanno inflitto sconfitte sempre pi\`u miserabili alle armate dell'umanit\`a. Non fosse stato per le magiche Pietre Guardiane installate lungo i confini orientale e meridionale della regione, i demoni avrebbero da tempo invaso il settentrione del vicino regno di Mendev e forse ancora oltre. La Quarta Crociata \`e andata via via spegnendosi per alcuni, ma \`e molto viva per diversi altri. Quel che \`e certo \`e che le paralizzanti carenze di risultati hanno abbattuto il morale dei crociati, mettendo a serio rischio l'intera guerra. E sebbene l'inquietante influenza dei demoni cominci a penetrare la mente di troppi governanti, impegnati in un'intestina e quasi paranoica caccia alle streghe, una striminzita minoranza di prelati e paladini non solo sostiene che l'ultima battaglia sia ancora in corso, ma che nell'immediato futuro sar\`a evidente il suo esito. La Quarta Crociata non pu\`o essere definita vivace, ma i crociati ci hanno azzeccato pi\`u di quanto pensino circa l'arrivo imminente di una svolta. Una svolta destinata ad essere a favore dell'Abisso.